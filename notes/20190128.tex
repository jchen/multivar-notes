% !TEX root = ../multivar-notes.tex
\subsubsection*{January 28, 2019}

\begin{proposition}
If $U\subset \R^n$ is open, and $\bm{f}:U\to\R^m$ is differentiable at $\bm{a}\in U$, then all directional derivatives of $\bm{f}$ at $\bm{a}$ exist, and the directional derivative in the direction $\vec{\bm{v}}$ is given by the formula

\begin{equation}
	[\bm{Df} (\bm{a})]\vec{\bm{v}} = \lim_{h\to 0}\frac{\bm{f}(\bm{a}+h\vec{\bm{v}})-\bm{f}(\bm{a})}{h}
\end{equation}
\end{proposition}

\begin{proof}
  \textit{Datailed in textbook p.130 (Proposition 1.7.14)}\end{proof}

\subsection{Rules for calculating derivatives}

(A lot of them are surprisingly similar to what we're used to seeing in Calculus BC!)
\begin{enumerate}[1.]
  \item If $\bm{f} : U \to \R^m$ is a constant function, then $\bm{f}$ is differentiable, and its derivative is $[0]$.
  \item If $\bm{f} : \R^n \to \R^m$ is linear, then it is differentiable everywhere, and its derivative at all points $\bm{a}$ is $\bm{f}$, \textit{i.e.}, $[\bm{Df}(\bm{a})]\vec{\bm{v}}=\bm{f}(\bm{v})$. 
  \item If $f_1, \dots, f_m : U \to \R$ are $m$ scalar valued functions differentiable at $\bm{a}$, then so is $\bm{f}=\Point{f_1 \\ \vdots \\ f_m}$ with derivative
\begin{equation}
	  [\bm{Df}(\bm{a})]\vec{\bm{v}}=\mtrx{[\bm{D}f_1(\bm{a})]\vec{\bm{v}} \\ \vdots \\ [\bm{D}f_m(\bm{a})]\vec{\bm{v}}}
\end{equation}
  The same applies the other direction, if $\bm{f}$ is differentiable with derivative $[\bm{Df}(\bm{a})]$, then its coordinate components $f_i$ are the $i$th row of the entire derivative matrix. 
  \item Given differentiable $\bm{f}, \bm{g}$ at $\bm{a}$, then
  \begin{equation}
  	[\bm{D}(\bm{f}+\bm{g})(\bm{a})] = [\bm{D}\bm{f}(\bm{a})]+[\bm{D}\bm{g}(\bm{a})]
  \end{equation}
  \item Given differentiable $f : U\to \R, \bm{g} : U\to \R^m$ at $\bm{a}$, then 
  \begin{equation}
  	[\bm{D}(f\bm{g})(\bm{a})]\vec{\bm{v}} = \underbrace{f(\bm{a})}_{\R} \underbrace{[\bm{D}\bm{g}(\bm{a})]\vec{\bm{v}}}_{\R^m}+\underbrace{([\bm{D}f(\bm{a})]\vec{\bm{v}})}_{\R}\underbrace{\bm{g}(\bm{a})}_{\R^m}
  \end{equation}
  \item Some variant of the quotient rule. This is too complicated\dots (see textbook p.138)
  \item Given differentiable $\bm{f} : U\to \R^m, \bm{g} : U\to \R^m$ at $\bm{a}$, then 
  \begin{equation}
  	[\bm{D}(\bm{f}\cdot\bm{g})(\bm{a})]\vec{\bm{v}} = \underbrace{\bm{f}(\bm{a})}_{\R^m} \cdot \underbrace{[\bm{D}\bm{g}(\bm{a})]\vec{\bm{v}}}_{\R^m}+\underbrace{[\bm{D}\bm{f}(\bm{a})]\vec{\bm{v}}}_{\R^m}\cdot \underbrace{\bm{g}(\bm{a})}_{\R^m}
  \end{equation}

\end{enumerate}

\begin{theorem}
  \textbf{(Chain rule).} Let $U\subset \R^n$, $V\subset \R^m$ be open sets, let $\bm{g} : U \to V$ and $\bm{f} : V \to \R^p$ be mappings, and let $\bm{a}$ be a point of $U$. If $\bm{g}$ is differentiable at $\bm{a}$ and $\bm{f}$ is differentiable at $\bm{g}(\bm{a})$, then the composition $\bm{f} \circ \bm{g}$ is differentiable at $\bm{a}$, and its derivative is given by
  \begin{equation}
  	\bm{D}[(\bm{f}\circ \bm{g})(\bm{a})]=[\bm{Df} (\bm{g}(\bm{a}))]\circ [\bm{Dg}(\bm{a})]
  \end{equation}
\end{theorem}
