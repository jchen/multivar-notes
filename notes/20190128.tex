% !TEX root = ../multivar-notes.tex
\lessondate{January 28, 2019}\\

\begin{proposition}
\textbf{(***)}
\end{proposition}

\begin{proof}
  Recall that the expression
  \[\bm{r}(\vec{\bm{h}})=\left(\bm{f}(\bm{a}+\vec{\bm{h}})-\bm{f}(\bm{a})\right)-[\bm{Df}(\bm{a})]\vec{\bm{h}}\]
  ***
\end{proof}

\ul{\textbf{Rules for calculating derivatives}}

(A lot of them are surprisingly similar)
\begin{enumerate}[1.]
  \item If $\bm{f} : U \to \R^m$ is a constant function, then $\bm{f}$ is differentiable, and its derivative is $[0]$.
  \item If $\bm{f} : \R^n \to \R^m$ is linear, then it is differentiable everywhere, and its derivative at all points $\bm{a}$ is $\bm{f}$, \textit{i.e.}, ***
  \item (differentiable just take singular derivative)
  \item sum
  \item product
  \item quotient rule
  \item composition
\end{enumerate}

\begin{theorem}
  \textbf{(Chain rule).} Let $U\subset \R^n$, $V\subset \R^m$ be open sets, let $\bm{g} : U \to V$ and $\bm{f} : V \to \R^p$ be mappings, and let $\bm{a}$ be a point of $U$. If $\bm{g}$ is differentiable at $\bm{a}$ and $\bm{f}$ is differentiable at $\bm{g}(\bm{a})$, then the composition $\bm{f} \circ \bm{g}$ is differentiable at $\bm{a}$, and its derivative is given by
  \[\bm{D}[(\bm{f}\circ \bm{g})(\bm{a})]=[\bm{Df} (\bm{g}(\bm{a}))]\circ [\bm{Dg}(\bm{a})]\]
\end{theorem}

\example
\textbf{(The derivative of a composition).} Define $\bm{g} : \R \to \R^3$ and $f : \R^3 \to \R$
\[f\Point{x \\ y \\ z}=x^2+y^2+z^2;\qquad \bm{g}(t)=\Point{t \\ t^2 \\ t^3}\]
