% !TEX root = ../multivar-notes.tex
\lessondate{February 18, 2019}

\ul{\textbf{Parametrization}}

Consider the unit circle in $\R^2$ (a manifold). 
\[x^2+y^2=1\]
A parametrization of this could be $\gamma(t)=\Point{\cos t \\ \sin t}, t\in (0,2\pi)$ \footnote{The open interval is of no significance, it goes away when integrating or working with it. }

\begin{defn}{Parametrization of a manifold}
A \ul{parametrization} of a $k$-manifold $M\subset \R^n$ is a mapping $\gamma : U \subset \R^k \to M$ satisfying the following conditions: 
	\begin{enumerate}[1.]
	\item $U$ is open. 
	\item $\gamma$ is $C^1$, one to one, and onto $M$. 
	\item $\left[\mathbf{D}\gamma (\bm{u})\right]$ is one to one for every $\bm{u}\in U$. 
	\end{enumerate}
\end{defn}

Example: 
\[\gamma : \Point{\theta \\ \varphi} \mapsto \Point{\cos \theta \cos \varphi \\ \sin \theta \cos \varphi \\ \sin \varphi}\]

We can visualize this in Mathematica using

\Mathematica{20190218}{1}{2}

\exercise{3.1.11}
\begin{enumerate}[a.]
	\item Find a parametrization for the union $X$ of the lines through the origin and a point of the parametrized curve $t\mapsto \Point{t \\ t^2 \\ t^3}$. 
	
	\answer{$t,u\mapsto \Point{ut \\ ut^2 \\ ut^3}$}
	
	\item Find an equation for the closure $\overline{X}$ of $X$. Is $\overline{X}$ exactly $X$? 
	
	\answer{$\overline{x} : \frac{y}{x} = \frac{z}{y}$ \textit{or} $y^2 = xz$}  This is the limit as $t$ approaches $\infty$ and there is one line that is never traced out. 
	
	\footnotesize
\Mathematica{20190218}{7}{8}
	\normalsize