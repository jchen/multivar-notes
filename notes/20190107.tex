% !TEX root = ../multivar-notes.tex
\section{Topology}
\subsection{Subsets of $\R$}
\subsubsection*{January 7, 2019}
We start with an in-depth exploration of topology first in single-dimensional reals.

\begin{defn}{Bounds}
Let $X\subseteq \R$. Then\dots
\begin{enumerate}
	\item $u\in \R$ is called an \ul{upper bound} of $X$ if $x\leq u,\ \forall x\in X$.
	\item $l\in \R$ is called a \ul{lower bound} of $X$ if $x\geq l,\ \forall x\in X$.
\end{enumerate}
\end{defn}

It is an axiomatic property of $\R$ that each subset of $\R$ bounded above has a least upper bound and, likewise, each subset that is bounded below has a greatest lower bound.

\begin{defn}{Extremum}
	Let $X\subseteq \R$ be bounded. Then\dots
	\begin{enumerate}
	\item $y = \Sup(X)$ (\ul{supremum} of X) if $y$ is an upper bound and, $y'$ is another upper bound, then $y' \geq y$.
	\item $z = \Inf(X)$ (\ul{infinum} of X) if $z$ is an lower bound and, $z'$ is another lower bound, then $z' \leq z$.
\end{enumerate}
Also if\dots
\begin{enumerate}
\item $\Sup(X)\in X$, then we call it the maximum of $X$.
\item $\Inf(X)\in X$, then we call it the minimum of $X$.
\end{enumerate}
\end{defn}

\example

$X = (0,1)$ \qquad
$\Sup(X) = 1$ \qquad
$\Inf(X) = 0$ \qquad
no max, no min

$X = [0,1]$ \qquad
$\Sup(X) = \Max(X) = 1$ \qquad
$\Inf(X) = \Min(X) = 0$

\begin{proposition}
	If $X\subseteq \R$, bounded above, then $y = \Sup(X)$ iff
	\begin{enumerate}[(i)]
		\item $y$ is an upper bound
		\item $\forall \epsilon > 0$, $\exists x\in X$ such that $x>y-\epsilon$
	\end{enumerate}
\end{proposition}
\begin{proof}
Let $y = \Sup(X)$. \begin{enumerate}[(i)]
\item is true by definition
\item Suppose $\exists \epsilon > 0$ such that there is no $x\in X$ with $x>y-\epsilon$. \\
Then $x\leq y-\epsilon \forall x\in X$. But that makes $y-\epsilon < y$ a smaller upper bound of $X$, which contradicts $y=\Sup(X)$
\end{enumerate}

Suppose next that (i) and (ii) hold for $y\in \R$. We show that $y = \Sup(X)$. Clearly, y is an upper bound by (i), so let $y'$ be a smaller upper bound for the sake of contradiction: $X\leq y' < y$ for all $x\in X$. Now consider $\epsilon = y-y'$. Then $y-\epsilon = y- (y-y') = y' \geq x \forall x\in X$. This contradicts (ii) because we have found an $\epsilon > 0$ such that $\not \exists x \in X$ greater than $y-\epsilon$.
\end{proof}

\begin{proposition}
	Let X be bounded below.
	\begin{equation}
		\Inf(X) = -\Sup(-X)
	\end{equation}
	where $-X = \{-x \mid x\in X\}$
\end{proposition}

\begin{proof}
	Let $y = \Sup (-X)$. Then $y \geq -x \implies -y \leq x$ for all $x\in X$, so $-y$ is a lower bound for $X$. Now assume for the sake of contradiction that $\exists -y' > -y$, another lower bound of $X$. Then $-y' \leq x \implies y' \geq -x$ for all $x\in X$. But $-y' > -y \implies y' < y$ so $y \neq \Sup(-X)$. Hence $\not \exists -y'$, another lower bound of $X$. $\implies -y = \Inf(X) \implies -\Sup(-X) = \Inf(X)$
\end{proof}

\begin{proposition}
	If A, B are bounded subsets of $\R$. Then $A\cup B$ is bounded and
	\begin{equation}
		\Sup(A\cup B) = \Max\Set{\Sup(A), \Sup(B)}
	\end{equation}
\end{proposition}
