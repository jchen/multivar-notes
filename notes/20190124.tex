% !TEX root = ../multivar-notes.tex
\subsection{Derivatives in $\R^n$}
\subsubsection*{January 22, 2019}


Let $f : \R^n \rightarrow \R^m$

Goal: local linearization of $f$ with error going to zero sufficiently fast.

\begin{defn}{Derivatives in $\R^n$}
Let $U\subset \R^n$ be an open subset and let $\bm{f} : U \to \R^m$ be a mapping; let $\bm{a}$ be a point in $U$. If there exists a linear transformation (represented by a matrix) $[\bm{Df}(\bm{x})] \in\mathcal{L}(\R^n, \R^m)$ such that
\begin{equation}
	\lim_{\vec{h}\rightarrow 0}\frac{1}{|\vec{h}|}(\bm{f}(\bm{x}+\vec{h})-\bm{f}(\vec{\bm{x}}))-[\bm{Df}(\bm{x})]\vec{h}=\vec{0}
\end{equation}
then $\bm{f}$ is \ul{differentiable} at $\bm{a}$, and $[\bm{Df}(\bm{x})]$ is unique and is the \ul{derivative} of $\bm{f}$ at $\bm{a}$. 
\end{defn}

If we know that $[\bm{Df}(\bm{x})]$ exists, we can calculate its matrix $[\bm{Jf}(\bm{x})]$ (Jacobian matrix) by evaluating $[\bm{Df}(\bm{x})]$ on the standard basis vectors.

We know that
\begin{align*}
0 &= \lim_{|h|\rightarrow 0}\frac{1}{|h\vec{e}_i|}\left( f(\bm{x}+h\vec{e}_i)-f(\bm{x})-[\bm{Df}(\bm{x})](h\vec{e}_i) \right) \\
&= \lim_{|h|\rightarrow 0}\frac{1}{|h|}\left( f(\bm{x}+h\vec{e}_i)-f(\bm{x})-h[\bm{Df}(\bm{x})](\vec{e}_i)\right) \\
&= \lim_{h\rightarrow 0}\frac{f(\bm{x}+h\vec{e}_i)-f(\bm{x})}{h} -[\bm{Df}(\bm{x})](\vec{e}_i) \\
\implies [\bm{Df}(\bm{x})]\vec{e}_i &= \lim_{h\rightarrow 0}\frac{f(\bm{x}+h\vec{e}_i)-f(\bm{x})}{h}\qquad (\#)
\end{align*}


\begin{defn}{Partial derivative}
The right-hand side of $\#$ is called the \ul{partial derivative} of $f$ (with respect to the $i$th variable evaluated at $\bm{x}$): 
\begin{equation}
	D_i f(\bm{x})\Def\lim_{h\rightarrow 0}\frac{1}{h} \left(f\Point{x_1  \\ \vdots\ \\ x_i + h \\ \vdots \\ x_n} - f\Point{x_1 \\ \vdots \\ x_i \\ \vdots \\ x_n}\right)
\end{equation}

(Given such limit exists, of course). Therefore, we can calculate it by considering $x_i$ the only variable, and holding all other components constant.

This limit is essentially the $i$th row in $[\bm{Df}]$ or $[\bm{Jf}]$. 
\end{defn}

There are a variety of notations for this derivative:
\begin{itemize}
\begin{minipage}{0.5\linewidth}
  \item $D_i f(x)$
  \item $D_x f(\bm{x}), D_y f(\bm{x}), D_z f(\bm{x})$
\end{minipage}\begin{minipage}{0.5\linewidth}
  \item $\frac{\delta f}{\delta x}$, $\frac{\delta f}{\delta x_2}$ \dots
  \item $f_x, f_y$ \dots
  \end{minipage}
\end{itemize}

Let's backtrack a bit for a full definition of the Jacobian: 
\begin{defn}{Jacobian matrix}
	The \ul{Jacobian matrix} of a function $\bm{f} : U\subset \R^n\to \R^m$ [i.e. $\bm{f}\Point{\bm{a}} = \Point{f_1(\bm{a}),\dots,f_n(\bm{a})}$] is the $m \times n$ matrix composed of the $n$ partial derivatives of $\bm{f}$ evaluated at $\bm{a}$: 
\begin{equation}
	[\bm{Jf}(\bm{a})] = \mtrx{\bm{J}f\Point{a_1 \\ \vdots \\ a_n}} \Def \mtrx{D_1f_1(\bm{a}) & \cdots & D_nf_1(\bm{a}) \\ \ \vdots & \ddots & \vdots \\ D_1f_m(\bm{a}) & \cdots & D_nf_m(\bm{a})}
\end{equation}
\end{defn}


\example
\begin{align*}
  f\Point{x \\ y} &= \sin{(x^2+y^3)} \\
  D_x f\Point{x \\ y} &= \cos{(x^2 + y^3)}\cdot 2x \\
  D_y f\Point{x \\ y} &= \cos{(x^2 + y^3)}\cdot 3y^2 \\
  \left[Df\Point{x \\ y}\right] &= \cos{(x^2 + y^3) [2x \quad 3y^2]}
\end{align*}

\textbf{Warning:} The Jacobian matrix is only the matrix of the derivative if the function is actually differentiable!

(\textbf{Preview:} We will see shortly that $f$ is differentiable if all its partials exist and are continuous. This gives us a pathway to prove whether a function is differentiable at a point, by manually proving that its partial derivatives match up to the Jacobian matrix.)

This $\bm{Df}$ gives us the rate of change in the axes, if we want to find the directional rate of change in any direction, we have to use a direction derivative.

\begin{defn}{Directional derivatives}
The \ul{directional derivative} of $\bm{f}$ at $\bm{x}$ in direction $\vec{\bm{v}}$ gives the rate of change of $f$ as we step into direction $\vec{\bm{v}}$. It is defined as
\begin{equation}
	\lim_{h\to 0}\frac{\bm{f}(\bm{x}+h\vec{\bm{v}})-\bm{f}(\bm{x})}{h}
\end{equation}
\end{defn}

We will see shortly that this evaluates to $[\bm{Df}(\bm{x})]\vec{\bm{v}}$ given the function is differentiable at $\bm{x}$. We can exploit this fact to prove differentiability at point $\bm{x}$ by showing that $[\bm{Df}(\bm{x})]\vec{\bm{v}}$ matches up to the directional derivative definition. 

\exercise{1.7.4} Using the definition, check whether the following functions are differentiable at 0.
\begin{enumerate}[a.]
  \item $f(x) = |x|^{3/2}$ \\
  \answer{Exists.}
  \item $\displaystyle f(x) = \begin{cases}x \cdot \ln{|x|} & \text{if } x \neq 0 \\ 0 & \text{if } x = 0 \end{cases}$ \\
  \answer{Does not exist.}
  \item $\displaystyle f(x) = \begin{cases}x/ \ln{|x|} & \text{if } x \neq 0 \\ 0 & \text{if } x = 0\end{cases}$ \\
  \answer{Exists.}
\end{enumerate}
