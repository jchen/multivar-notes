% !TEX root = ../multivar-notes.tex
\lessondate{January 22, 2019}\\

\ul{\textbf{Derivatives} (continued)}

Let $f : \R^n \rightarrow \R^m$

Goal: local linearization of $f$ with error going to zero sufficiently fast.

\[\lim_{\vec{h}\rightarrow 0}\frac{1}{|\vec{h}|}(f(\bm{x}+\vec{h})-f(\vec{\bm{x}}))-[\bm{Df}(\bm{x})]\vec{h}=\vec{0}\qquad [\bm{Df}(\bm{x})]\in L(\R^n, \R^m)\]

If we know that $[\bm{Df}(\bm{x})]$ exists, we can calculate its matrix $[\bm{Jf}(\bm{x})]$ (Javobian matrix) by evaluating $[\bm{Df}(\bm{x})]$ on the standard basis vectors.

We know that
\begin{align*}
0 &= \lim_{|h|\rightarrow 0}\frac{1}{|h\vec{e}_i|}\left( f(\bm{x}+h\vec{e}_i)-f(\bm{x})-[\bm{Df}(\bm{x})](h\vec{e}_i) \right) \\
&= \lim_{|h|\rightarrow 0}\frac{1}{|h|}\left( f(\bm{x}+h\vec{e}_i)-f(\bm{x})-h[\bm{Df}(\bm{x})](\vec{e}_i)\right) \\
&= \lim_{h\rightarrow 0}\frac{f(\bm{x}+h\vec{e}_i)-f(\bm{x})}{h} -[\bm{Df}(\bm{x})](\vec{e}_i) \\
\end{align*}

\vspace{-12pt}

\[\implies [\bm{Df}(\bm{x})]\vec{e}_i = \lim_{h\rightarrow 0}\frac{f(\bm{x}+h\vec{e}_i)-f(\bm{x})}{h}\qquad (\#)\]

The right-hand side of $\#$ is called the partial derivative of $f$ at $x$: in components, it looks as the follows:
\[\lim_{h\rightarrow 0}\frac{
f\Point{x_1& \\ x_2& \\ \vdots& \\ x_i &+ h \\ \vdots & \\ x_n&} - f\Point{x_1 \\ x_2 \\ \vdots \\ x_i \\ \vdots \\ x_n}
}{h}\]

Therefore, we can calculate it by considerinf $x_i$ the only variable, and holding all other components constant.

There are a variety of notations for this derivative:
\begin{itemize}
  \item $D_i f(x)$
  \item $D_x f(\bm{x}), D_y f(\bm{x}), D_z f(\bm{x})$
  \item $\frac{\delta f}{\delta x}$, $\frac{\delta f}{\delta x_2}$ \dots
  \item $f_x, f_y$ \dots
\end{itemize}

\example
\begin{align*}
  f\Point{x \\ y} &= \sin{(x^2+y^3)} \\
  D_x f\Point{x \\ y} &= \cos{(x^2 + y^3)}\cdot 2x \\
  D_y f\Point{x \\ y} &= \cos{(x^2 + y^3)}\cdot 3y^2 \\
  \left[Df\Point{x \\ y}\right] &= \cos{(x^2 + y^3) [2x \quad 3y^2]}
\end{align*}

\textbf{Warning:} The Jacobian matrix is only the matrix of the derivative if the function is actually differentiable!

(\textbf{Preview:} We will see shortly that $f$ is differentiable if all its partials exist and are continuous.)

This $Df$ gives us the rate of change in the axes, if we want to find the directional rate of change in any direction, we have to use a direction derivative.

\begin{defn}
The \ul{directional derivative} of $f$ at $x$ in direction \vec{v} gives the rate of change of $f$ as we step into direction $\vec{v}$. It is defined as
\[\lim_{h\to 0}\frac{f(x+h\vec{v})-f(x)}{h}\]
We will see shortly that this evaluates to $[Df(x)]\vec{v}$.
\end{defn}

Let's brush up on some simple derivatives first:
\begin{enumerate}[a.]
  \item $f(x) = \sin^3{(x^2 + \cos{x})} = \sin$
  \item ***
\end{enumerate}

\exercise{1.7.4} Using the definition, check whether the following functions are differentiable at 0.
\begin{enumerate}[a.]
  \item $f(x) = |x|^{3/2}$
  \item $\displaystyle f(x) = \begin{cases}x \cdot \ln{|x|} & \text{if } x \neq 0 \\ 0 & \text{if } x = 0 \end{cases}$ \\
  \answer{Does not exist.}
  \item $\displaystyle f(x) = \begin{cases}x/ \ln{|x|} & \text{if } x \neq 0 \\ 0 & \text{if } x = 0\end{cases}$ \\
  \answer{Exists.}
\end{enumerate}
