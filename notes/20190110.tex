% !TEX root = ../multivar-notes.tex
\lessondate{January 10, 2019}\\

Recall \textbf{Prop 1.5}: 	If A, B are bounded subsets of $\R$. Then $A\cup B$ is bounded and
\[\Sup(A\cup B) = \Max\Set{\Sup(A), \Sup(B)}\]

\begin{proof}
  \begin{enumerate}[1]
    \item Show that $x \leq \Max\Set{\Sup(A), \Sup(B)}$ for all $x\in A\cup B$ \\xr
    Case 1: $x\in A \implies x \leq \Sup(A) \leq \Max\Set{\Sup(A),\Sup(B)}$ \\
    Case 2: $x\in B \implies x \leq \Sup(B) \leq \Max\Set{\Sup(A),\Sup(B)}$

    \item Take $\epsilon > 0$ and consider $\Max\Set{\Sup(a), \Sup(B)}-\epsilon$ \\
    Case 1: $\Max\Set{\Sup(A), \Sup(B)}=\Sup A \implies \exists x\in A$ such that $x>\Sup(A)-\epsilon \implies x \in A\cup B $ such that $x>\Max\Set{\Sup(A), \Sup(B)}-\epsilon$ \\
    Case 2: $\Max\Set{\Sup(A), \Sup(B)}=\Sup B \implies $ \textit{left to the reader, follows similarly as above. }
  \end{enumerate}
\end{proof}

Also recall\dots
\exercise{1.5.5}
For each of the following subsets of $\R$ and $\R^2$, state whether it is open or closed (or both or neither), and prove it.
\begin{enumerate}[a.]
	\item $\Set{\Point{x \\ y} \in \R^2 \mid 1 < x^2+y^2 < 2}$ \\
  \answer{Open.} \\
  \begin{proof}
		Let $p\in A$(nnulus). $1 < |p-0| < \sqrt{2}$. To show: $\exists \epsilon > 0$ s.t. all points in $B_\epsilon (p)$ are between $1$ and $\sqrt{2}$ from 0. There is such $\epsilon$, specifically
		\[\epsilon = \frac{1}{2}\cdot \Min(\sqrt{2}-\left|p\right|, \left|p\right| - 1)\]
    Now we show that for $x\in B_\epsilon (p)$, $1 < |x|^2 < 2$: \\
    WLOG: Consider $p\in (1,\sqrt{2})$ on the $x$-axis. Then the neighborhood of $p$ is:
    \[B_\epsilon (p) = \Set{\Point{p+r \sin{\theta} \\ r \sin{\theta}} \,\middle\vert\, r\in [0,\epsilon)}\]
    \begin{align*}
      \left| \Point{p+r \sin{\theta} \\ r \sin{\theta}} \right|^2 & = p^2 + 2pr \cos{\theta} + r^2\cos^2 \theta + r^2 \sin^2 \theta \\
      &= p^2 + 2pr \cos{\theta} + r^2
    \end{align*}
    \[(p-r)^2 = p^2 - 2pr + r^2 \leq p^2 + 2pr \cos{\theta} + r^2 \leq p^2 + 2pr+r^2 = (p+r)^2\]
    \[\text{Since }r  < (\sqrt{2}-p), (p+r)^2 < (p+\sqrt{2}-p)^2 = 2\]
    \[\text{Also since } r < (p-1), (p-r)^2 > \left(p-(p-1)\right)^2 = 1\]
	\end{proof}

We could also use the triangle inequality ($|a+b| \leq |a| + |b|$ and $|a-b|\geq ||a|-|b||$):
\[|p+r| \leq |p| + |r| < |p| + (\sqrt{2} - |p|) = \sqrt{2}\]
\[|p-r| \geq |p| - |r| > |p| - (|p| - 1) = 1\]

	\item $\Set{\Point{x \\ y} \in \R^2 \mid xy\neq 0}$ \\
  \answer{Open.} \\
	\begin{proof}
    Consider $B_\epsilon (p)$ with $\epsilon = \frac{1}{2} min\Set{|x|, |y|}$.
	\end{proof}
	\item $\Set{\Point{x \\ y} \in \R^2 \mid y=0}$ \\
  \answer{Closed.} \\
  \begin{proof} Consider the complement, $\Set{\Point{x \\ y} \in \R^2 \mid y\neq 0}$. Following a similar logic as \textit{b}, consider $\epsilon = \frac{x}{2}$.  \end{proof}
	\item $\Set{\mathbb{Q} \subset \R}\qquad$ (the rational numbers) \\
  \answer{Neither.}
\end{enumerate}

\exercise{1.5.3} Prove the following statements for open subsets of $\R^n$:
\begin{enumerate}[a.]
  \item \textbf{Any union of open sets is open.} \\
  \begin{proof}
    Let $X_i$, $i \in I$, be open. Consider $Y = \bigcup_{i\in I}X_i$. \\
    To show: each $y \in Y$ is an interior point of $Y$. \\
    Let $y\in Y$ belong to arbitrary $X_i$, for some $i\in I$. As $X_i$ is open, $y$ is also an interior point of $X_i$. So $\exists \epsilon > 0$ s.t. $B_\epsilon (y) \subset X_i \subseteq Y$ $\implies y$ is an interior point of $Y$.
  \end{proof}
  \item \textbf{A finite intersection of open sets is open.} \\
  \begin{proof}
    Consider $Z = \bigcap^n_{i=I}X_i$. \\
    To show: each $z\in Z$ is an interior point of $Z$. Since $z\in Z$, $z\in X_i$ for $i=1,\dots,n$. Since $X_i$ is open, $\exists\epsilon_i > 0 \mid B_{\epsilon_i}(z) \subset X_i$. As there are finitely many $i$, we choose the smallest $\epsilon=min\Set{\epsilon_i \mid i = 1,\dots,n}$. Then we have
    \[B_\epsilon (z) \subset B_{\epsilon_i}(z) \subset X_i \text{ for all } x=1,\dots,n\]
    Thus $B_\epsilon (z)\subset Z$, making $z$ an interior point of $Z$.
  \end{proof}

  \item\textbf{An infinite interesection of open sets is not necessarily open.}
  \begin{proof}
    \[\bigcap_{n=1}^{\infty} \Set{x \,\middle\vert\, x\in \left(-\frac{1}{n}, \frac{1}{n}\right)}=\Set{0}\]
  \end{proof}
\end{enumerate}
