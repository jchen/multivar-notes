% !TEX root = ../multivar-notes.tex
\subsection{Tangent Spaces}
\subsubsection*{February 21, 2019}


As we saw, a $k$-manifold in $\R^n$ can be thought of as the solution to an equation $F(x)=0$ or the image of a parametrization $\gamma : \R^k \to \R^{n}$.

Since both $F$ and $\gamma$ are assumed to be $C^1$, their derivatives give a local linear approximation of the manifold which we call the ``tangent space''.

The linear equivalent of ``$F(x)=0$'' is $ker[\bm{D}F(x)]$, and the linear equivalent of ``$im\ \gamma(u)$'' is ``$im[\bm{D}\gamma(u)]$''.

The tangent space at a point $\Point{\bm{x}_0 \\ \bm{y}_0}\in M$ is given by equation
\begin{equation}
	\underbrace{\bm{y}-\bm{y}_0}_{\text{change in output}} = \mtrx{\bm{Df}(\bm{x}_0)}\underbrace{\bm{x}-\bm{x}_0}_{\text{change in input}}
\end{equation}

We use $\dot{\bm{x}}$ and $\dot{\bm{y}}$ to denote \ul{increments} to $\bm{x}$ and $\bm{y}$ respectively, so $\dot{\bm{x}} = \bm{x}-\bm{x}_0$ and $\dot{\bm{y}} = \bm{y}-\bm{y}_0$. 

Formally\dots 
\begin{defn}{Tangent space to a manifold}
	Let $M\in \R^n$ be a $k$-dimensional manifold. The \ul{tangent space} to $M$ at $\bm{z}_0\Def\Point{\bm{x}_0 \\ \bm{y}_0}$, denoted $T_{\bm{z}_0}M$, is the graph of the linear transformation $[\bm{Df}(\bm{x}_0)]$. 
\end{defn}

Naturally, this prompts the question...how do we actually compute the tangent spaces to manifolds. If a manifold is defined by an equation, we can see it as the null space of the derivative of a function $\bm{F}(\bm{z}) = \bm{0}$. 
\begin{theorem}
	\textbf{(Tangent space to a manifold given by equation).} If $\bm{F}(\bm{z}) = \bm{0}$ describes a manifold $M$, and $\mtrx{\bm{DF}(\bm{z}_0)}$ is onto for some $\bm{z}_0\in M$, then the tangent space $T_{\bm{z}_0}M$ is the kernel of $\mtrx{\bm{DF}(\bm{z}_0)}$:
	\begin{equation}
		T_{\bm{z}_0}M = ker \mtrx{\bm{DF}(\bm{z}_0)}
	\end{equation}
\end{theorem}

\begin{rem}
We might be used to seeing functions in the form $\bm{f}(\bm{x}) = \bm{y}$, which we have to ensure we rewrite as $\bm{F}(\bm{x},\bm{y})=\bm{f}(\bm{x})-\bm{y}=0$ in order to find the tangent space this way. 	
\end{rem}

\begin{theorem}
	\textbf{(Tangent space of manifold given by parametrization).} Let $U\in \R^k$ be open, and let $\gamma : U\to \R^n$ be a parametrization of a manifold $M$. Then
	\begin{equation}
		T_{\gamma(\bm{u})}M = im\mtrx{\bm{D}\gamma (\bm{u})}
	\end{equation}
\end{theorem}

\example Tangent spaces to the unit circle
\begin{enumerate}[1.]
\item
\begin{minipage}{0.7\textwidth}
\[F\Point{x \\ y}=x^2+y^2-1\]
\[\bm{D}F\Point{x \\ y}=\mtrx{2x & 2y}\]
\[ker\mtrx{2x & 2y} = \Set{\mtrx{\dot{x} \\ \dot{y}}\in \R^2 \bigm| \mtrx{2x & 2y}\mtrx{\dot{x} \\ \dot{y}}=0}\]
\end{minipage}
\begin{minipage}{0.15\textwidth}
If $\Point{x \\ y} = \Point{1 \\ 0}$,
\begin{align*}
  \mtrx{2 & 0}\mtrx{\dot{x} \\ \dot{y}} &= 0 \\
  \implies 2\dot{x} &= 0 \\
  \implies \dot{x} &= 0
\end{align*}
\end{minipage}

\begin{center}
\begin{tikzpicture}
\newcommand{\myangle}{0}
\tkzInit[xmin=-2,xmax=2,xstep=1,ymin=-2,ymax=2,ystep=1]
\tkzDrawX \tkzDrawY
\tkzDefPoint(0,0){c}
\tkzDefPoint(1,1){a0}
\tkzRadius=1 cm
\tkzDrawCircle[R,thick,color=Cerulean](c,\tkzRadius)
\tkzDefPointBy[rotation=center c angle \myangle](a0)
\tkzGetPoint{a}
\tkzTangent[from with R = a](c,\tkzRadius)
\tkzGetPoints{e}{f}
\tkzDrawLine[add = 1 and 2,color=OrangeRed,thick](a,f)
\tkzDrawSegment(c,f)
\tkzDrawPoints[size=3,fill](f,c)
\end{tikzpicture}
\end{center}

The \ul{tangent line} will be the kernel of $\bm{D}F$ w.r.t the $x$-$y$-coordinate system:
\[\boxed{x-1=0}\]

If $\Point{x \\ y} = \Point{\sqrt{\frac{1}{2}} \\ \sqrt{\frac{1}{2}}}$, then $ker \bm{D}F = ker\mtrx{\sqrt{2} & \sqrt{2}}$, so we solve
\begin{align*}
\sqrt{2}\dot{x}+\sqrt{2}\dot{y}&=0 \\
\dot{y}&=-\dot{x} & \text{(tangent space)} \\
\System{y-\sqrt{\frac{1}{2}}} &=-\System{x-\sqrt{\frac{1}{2}}} & \text{(tangent line)}
\end{align*}

\begin{center}
\begin{tikzpicture}
\newcommand{\myangle}{45}
\tkzInit[xmin=-2,xmax=2,xstep=1,ymin=-2,ymax=2,ystep=1]
\tkzDrawX \tkzDrawY
\tkzDefPoint(0,0){c}
\tkzDefPoint(1,1){a0}
\tkzRadius=1 cm
\tkzDrawCircle[R,thick,color=Cerulean](c,\tkzRadius)
\tkzDefPointBy[rotation=center c angle \myangle](a0)
\tkzGetPoint{a}
\tkzTangent[from with R = a](c,\tkzRadius)
\tkzGetPoints{e}{f}
\tkzDrawLine[add = 1 and 2,color=OrangeRed,thick](a,f)
\tkzDrawSegment(c,f)
\tkzDrawPoints[size=3,fill](f,c)
\end{tikzpicture}
\end{center}


In general, for $\Point{x \\ y}$ on the circle, the tangent space to the circle at point $\Point{x \\ y}$ is given by
\[ker\mtrx{2x & 2y}\]
\[=\Set{\mtrx{\dot{x} \\ \dot{y}} \bigm| x\dot{x}+y\dot{y}=0}\]

\item A parametrization of the unit circle is given by
\[\gamma(t)=\Point{\cos t \\ \sin t},t\in (0,2\pi)\]
\[\bm{D}\gamma(t)=\mtrx{-\sin t \\ \cos t}\]
Each $t\in(0,2\pi)$ gives a point on the circle with the tangent space spanned by $\mtrx{-\sin t \\ \cos t}$.

$t=0 : \bm{D}\gamma(0)=\mtrx{0 \\ 1}$

\begin{center}
\begin{tikzpicture}
\newcommand{\myangle}{0}
\tkzInit[xmin=-2,xmax=2,xstep=1,ymin=-2,ymax=2,ystep=1]
\tkzDrawX \tkzDrawY
\tkzDefPoint(0,0){c}
\tkzDefPoint(1,1){a0}
\tkzRadius=1 cm
\tkzDrawCircle[R,thick,color=Cerulean](c,\tkzRadius)
\tkzDefPointBy[rotation=center c angle \myangle](a0)
\tkzGetPoint{a}
\tkzTangent[from with R = a](c,\tkzRadius)
\tkzGetPoints{e}{f}
\tkzDrawLine[add = 1 and 2,color=OrangeRed,thick](a,f)
\tkzDrawSegment(c,f)
\tkzDrawPoints[size=3,fill](f,c)
\end{tikzpicture}
\end{center}

$\displaystyle t=\frac{\pi}{4} : [\bm{D}\gamma(\frac{\pi}{4})]=\mtrx{-\frac{\sqrt{2}}{2} \\ \frac{\sqrt{2}}{2}}$

$\qquad \implies span\System{\mtrx{-\frac{\sqrt{2}}{2} \\ \frac{\sqrt{2}}{2}}}=span\System{\mtrx{-1 \\ 1}}$

\begin{center}
\begin{tikzpicture}
\newcommand{\myangle}{45}
\tkzInit[xmin=-2,xmax=2,xstep=1,ymin=-2,ymax=2,ystep=1]
\tkzDrawX \tkzDrawY
\tkzDefPoint(0,0){c}
\tkzDefPoint(1,1){a0}
\tkzRadius=1 cm
\tkzDrawCircle[R,thick,color=Cerulean](c,\tkzRadius)
\tkzDefPointBy[rotation=center c angle \myangle](a0)
\tkzGetPoint{a}
\tkzTangent[from with R = a](c,\tkzRadius)
\tkzGetPoints{e}{f}
\tkzDrawLine[add = 1 and 2,color=OrangeRed,thick](a,f)
\tkzDrawSegment(c,f)
\tkzDrawPoints[size=3,fill](f,c)
\end{tikzpicture}
\end{center}

\end{enumerate}

\begin{defn}{Tangent space to a manifold}
***
\end{defn}

\begin{theorem}
  \textbf{(Tangent space to a manifold given by equations).} If $\bm{F}(z)=\bm{0}$ describes ***
\end{theorem}

\begin{proposition}
  \textbf{(Tangent space of a manifold given by parametrization).} ***
\end{proposition}

\exercise{3.2.4}
For each of the following functions $f$ and points $\Point{a \\ b}$, state whether there is a tangent plane to the graph of $f$ at the point $\Point{a \\ b \\ f\Point{a \\ b}}$. If there is such a tangent plane, find its equation, and compute the intersection of the tangent plane with the graph.

\begin{enumerate}[a.]
  \item $\displaystyle f\Point{x \\ y} = x^2 - y^2$ at $\Point{1 \\ 1}$

\Mathematica{20190221}{13}{14}

  The graph of $z=f\Point{x \\ y}$ is parametrized by
  \[\gamma\Point{u \\ v}=\Point{u \\ v \\ f(u,v)}=\Point{u \\ v \\ u^2-v^2}\]
  \[\bm{D}\gamma\Point{u \\ y}=\mtrx{1 & 0 \\ 0 & 1 \\ 2u & -2v}\]
  \[\bm{D}\gamma\Point{1 \\ 1}=\mtrx{1 & 0 \\ 0 & 1 \\ 2 & -2}\]
  \[im \left[\bm{D}\gamma\Point{1 \\ 1}\right] = \Set{\Point{\dot{x} \\ \dot{y} \\ \dot{z}} \bigm| \dot{x} = s, \dot{y} = t, \dot{z}=2s-2t}\]
  \[tangent\ plane = \Set{\Point{\dot{x} \\ \dot{y} \\ \dot{z}} \bigm| \dot{x} = 1 + s, \dot{y} = 1 + t, \dot{z}=2s-2t}\]

  We can check this using ***
  
  This is the tangent plane graphed: 
  
  \Mathematica{20190221}{15}{16}

  \item $\displaystyle f\Point{x \\ y} = \sqrt{x^2+y^2}$ at $\Point{0 \\ 0}$
  
  \Mathematica{20190221}{17}{18}
  \[F\Point{x \\ y \\ z} = z^2 - x^2 - y^2 = 0\]
  \[\mtrx{\bm{D}F\Point{x \\ y \\ z}}=\mtrx{-2x & -2y & 2z}\]
  \[\mtrx{\bm{D}F\Point{0 \\ 0 \\ 0}}=\mtrx{0 & 0 & 0}\]
  Evidently, this matrix is not onto so there does not exist a tangent plane at point $\Point{0 \\ 0 \\ 0}$.

  \item Same function as above but at point $\Point{1 \\ -1}$

  \[\mtrx{\bm{D}F\Point{x \\ y \\ z}}=\mtrx{-2x & -2y & 2z}\]
  \[\mtrx{\bm{D}F\Point{1 \\ -1 \\ \sqrt{2}}}=\mtrx{-2 & 2 & 2\sqrt{2}}\]
  \[ker \mtrx{\bm{D}F\Point{1 \\ -1 \\ \sqrt{2}}}=\Set{\Point{\dot{x} \\ \dot{y} \\ \dot{z}} \bigm| -2\dot{x} +2\dot{y} +2\sqrt{2}\dot{z}=0}\]
  \[-\dot{x}+\dot{y}+\sqrt{2}\dot{z}=0\]
  \[-(x-1)+(y+1)+\sqrt{2}(z-\sqrt{2})=0\]
  \[tangent\ plane : -x + y + \sqrt{2}z = 0\]

\end{enumerate}
