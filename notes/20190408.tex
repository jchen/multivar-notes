% !TEX root = ../multivar-notes.tex
\subsubsection*{April 8, 2019}

Content to be covered on the test: 
\begin{itemize}
	\item Riemann Integrals
	\item Fubini's Theorem
	\item Change of Coordinates
\end{itemize}

\section{Integrals (but more abstract)}
\subsection{Orientation}
Orientation gives us a sense of sign or direction of vector spaces. For example, when calculating flux through a surface, one side is the preferred direction (so flux through that side is positive), and likewise the other direction will be negative. 
\begin{defn}{Orientation of vector space}
	Let $V$ be a finite-dimensional real vector space, and let $\mathcal{B}_V$ be the set of bases of $V$. An \ul{orientation} of $V$ is a map $\Omega : \mathcal{B}_V \to \{+1, -1\}$ such that if $\{\bm{v}\}$ and $\{\bm{v'}\}$ are two bases with change of basis matrix $\mtrx{P_{\bm{v}\to \bm{v'}}}$, then
	\begin{equation}
		\Omega\left(\{\bm{v'}\}\right) = \mathrm{sgn}\System{\Det \mtrx{P_{\bm{v'}\to \bm{v}}}}\Omega\left(\{\bm{v}\}\right)
	\end{equation}
	A basis $\{\bm{vw}\}\in\mathcal{B}_V$ is called \ul{direct} if $\Omega\left(\{\bm{w}\}\right)=+1$; it is called \ul{indirect} if $\Omega\left(\{\bm{w}\}\right)=-1$. 
\end{defn}

\example 
\[\System{\mtrx{1\\0}, \mtrx{0\\1}}\to \System{\mtrx{2\\0}, \mtrx{0\\2}}\]
\[\Det \mtrx{2 & 0 \\ 0 & 2} > 0\]

\begin{defn}{Orientation of manifold}
	An \ul{oriantation} of a $k$-dimensional manifold $M\subset \R^n$ is a continuous map $\mathcal{B}(M)\to \{+1,-1\}$ whose restriction to each $\mathcal{B}_{\bm{x}}M$ is an orientation of $T_{\bm{x}}M$. 
\end{defn}

\begin{proposition} $ $ *** \\[-12pt]
	\begin{enumerate}
		\item \bm{Orienting poitns.}
		\item \bm{Orienting open subsets of $\R^n$.}
		\item \bm{Orienting a curve.}
		\begin{equation}
			\Omega_{\bm{x}}^{\vec{\bm{t}}} (\vec{\bm{v}}) \Def \mathrm{sgn} (\vec{\bm{t}}(\bm{x})\cdot \vec{\bm{v}})
		\end{equation}
		\item \bm{Orienting a surface in $\R^3$.} (In fact, a hyper-surface in $\R^n$)
		\item 
	\end{enumerate}
\end{proposition}

\exercise{6.3.4} Find a vector field that orients the curve given by $x+x^2+y^2=2$. 

\[x+x^2+y^2-2=0\]
