% !TEX root = ../multivar-notes.tex
\lessondate{January 14, 2019}

\begin{defn}{Convergent sequence; limit of sequence}
A sequence $i \mapsto \bm{a}_i$ if points in $\R^n$ \ul{converges} to $\bm{a}\in \R^n$ if
\[\forall \epsilon > 0,\ \exists M \text{ s.t. } m > M \implies |\bm{a}_m-\bm{a}| < \epsilon\]
We then call $\bm{a}$ the \ul{limit} of the sequence.
\end{defn}

\begin{proposition}
  \textbf{(Convergence in terms of coordinates).} A sequence $m\mapsto \bm{a}_m$ with $\bm{a}_m\in\R^n$ converges to $\bm{a}$ if and only if each coordinate converges; i.e., if for all $j$ with $1\leq j\leq n$, the $j$th coordinate of $\bm{a}_m$ converges to $\bm{a}_j$, the $j$th coordinate of the limit $\bm{a}$.
\end{proposition}

\begin{proof}
(p.88) The gist of the proof is to find sufficiently large $M$ for given $\epsilon$, in this case we set $M = max\Set{M_i}$ which guarantees that we stay within the error. 
\end{proof}

\begin{proposition}
\textbf{(Limit of sequence is unique). } If the sequence $i\mapsto \bm{a}_i$ of points in $\R^n$ converges to $\bm{a}$ and to $\bm{b}$, then $\bm{a}=\bm{b}$.
\end{proposition}

\begin{proof}
Let the sequence $i\mapsto \bm{a}_i$ converge to both $\bm{a}$ and $\bm{b}$. Then \[\forall \epsilon > 0, \exists M_a \land M_b \text{ s.t. } m > M_a, m > M_b \implies |\bm{a}-\bm{a}_{m}| < \frac{\epsilon}{2} \land |\bm{a}_{m} - \bm{b}| < \frac{\epsilon}{2}\]
\[|\bm{a}-\bm{b}|=|(\bm{a}-\bm{a}_m)+(\bm{b}_m-\bm{b})| \leq |\bm{a}-\bm{a}_{m}|+ |\bm{a}_{m} - \bm{b}| = \epsilon\]
\[\implies |\bm{a}-\bm{b}| = 0 \implies \bm{a}=\bm{b}\]
\end{proof}

\begin{theorem}
\textbf{(The arithmetic of limits of sequences). } All arithmetics that apply to limits apply here.
\end{theorem}

\begin{proposition}
\textbf{(Sequence in closed set).}
\begin{enumerate}
  \item Let $i\mapsto \bm{x}_i$ be a sequence in a closed set $C \subset \R^n$ converging to $\bm{x}_0\in \R^n$. Then $\bm{x}_0\in C$.
  \item Conversely, if every convergent sequence in a set $C \in \R^n$ converges to a point in $C$, then $C$ is closed. 
\end{enumerate}
\end{proposition}
