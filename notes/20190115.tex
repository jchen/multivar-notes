% !TEX root = ../multivar-notes.tex
\lessondate{January 15, 2019}\\

\begin{defn}{Limit of a function}
Let $X$ be a subset of $\R^n$ and $\bm{x}_0$ a point in $\overline{X}$ (note $\overline{X}=X\cup \delta X$). A function $\bm{f} : X \rightarrow \R^m$ has the \ul{limit} $\bm{a}$ at $\bm{x}_0$:
\[\lim_{\bm{x}\rightarrow \bm{x}_0}\bm{f}(\bm{x}) = \bm{a}\]
if $\forall \epsilon > 0, \exists\delta > 0$ s.t. $\forall\bm{x}\in X$,
\[|\bm{x}-\bm{x}_0|<\delta \implies |\bm{f}(\bm{x})-\bm{a}|<\epsilon\]

Related Prop: \emph{If a function has a limit, it is unique. }
\end{defn}

\begin{proposition}
\textbf{(Convergence by coordinates).} Suppose
\[U\subset \R^n, \quad \bm{f}=\Point{f_1 \\ \vdots \\ f_m}: U\rightarrow \R^m,\text{ and }\bm{x}_0\in\overline{U}\]
Then
\[\lim_{\bm{x}\to\bm{x}_0} \bm{f}(\bm{x})=\Point{a_1 \\ \vdots \\ a_m}\text{ iff }\lim_{\bm{x}\to\bm{x}_0} f_i(\bm{x})=a_i, i=1,\dots,m.\]
\end{proposition}

The above proposition basically states that for a multi-dimensional function, with each coordinate a function that has a limit, the limit of the multi-dimensional function is simply the individual limits as its coordinates. 

\begin{theorem}
  \textbf{(Limits of functions).} The same rules for traditional limits apply: addition, multiplication, division. Additional rules are as follows:
  \begin{enumerate}
    \item The dot product
    \[\lim_{\bm{x}\to\bm{x}_0}(\bm{f} \cdot \bm{g})(\bm{x})=\lim_{\bm{x}\to\bm{x}_0}\bm{f}(\bm{x})\cdot \lim_{\bm{x}\to\bm{x}_0}\bm{g}(\bm{x})\]
    \item The limit of the product of two functions, one of whose limit evaluates to $0$ and another which is bounded, will be $0$ (see textbook p.95, there are nuances to this rule).
  \end{enumerate}
\end{theorem}

\exercise{1.5.14} State whether the following limits exist, and prove it.
\begin{enumerate}[a.]
  \item $\displaystyle \lim_{(x,y)\rightarrow (1,2)} \frac{x^2}{x+y}$\\
  \answer{Exists.} We can simply evaluate the function at the given point. The polynomial and non-diminishing quotient nature of the function guarantee its existence. 
  
  \item $\displaystyle \lim_{(x,y)\rightarrow (0,0)} \frac{\sqrt{|x|}y}{x^2+y^2}$\\
  \answer{Does not exist.} It intuitively makes sense as the power in the denominator outweigh the power in the numerator. We can prove this by approaching this function and showing that it is unbounded. Let us approach this from $y=x$: 
  \[\lim_{y,x\to 0}\frac{\sqrt{|x|}x}{x^2+x^2}=\lim_{y,x\to 0}\frac{x^{3/2}}{2x^2}=\lim_{y,x\to 0}\frac{1}{2x^{1/2}}=\infty\quad (!)\]

\end{enumerate}
