% !TEX root = ../multivar-notes.tex
\lessondate{February 26, 2019}

\exercise{3.2.6}
\begin{enumerate}
\item Show that the subset $X\subset \R^4$ where
\[x_1^2+x_2^2-x_3^2-x_4^2=0\text{ and }x_1+2x_2+3x_3+4x_4=4\]
is a manifold in $\R^4$ in a neighborhood of the point $\bm{p}=\Point{1 \\ 0 \\ 1 \\ 0}$. 

\begin{proof}
\begin{align*}
F_1 &= x_1^2+x_2^2-x_3^2-x_4^2 \\
F_2 &= x_1+2x_2+3x_3+4x_4 -4
\end{align*}
\[F\Point{x_1 \\ x_2 \\ x_3 \\ x_4}=\Point{x_1^2+x_2^2-x_3^2-x_4^2 \\ x_1+2x_2+3x_3+4x_4 -4} = \Point{0 \\ 0}\]
\[\bm{D}F\Point{x_1 \\ x_2 \\ x_3 \\ x_4} = \mtrx{2x_1 & 2x_2 & 2x_3 & 2x_4 \\ 1 & 2 & 3 & 4}\]
\[\bm{D}F\Point{1 \\ 0 \\ 1 \\ 0} = \mtrx{2 & 0 & -2 & 0 \\ 1 & 2 & 3 & 4}\]
Which are linearly independent, which shows that this is onto, and so it is a smooth manifold locally at $\Point{1 \\ 0 \\ 1 \\ 0}$. 
\end{proof}

\item What is the tangent space to $X$ at $\bm{p}$? 

\[T_pM = ker\mtrx{2 & 0 & -2 & 0 \\ 1 & 2 & 3 & 4} \overset{rref}{=}\mtrx{1 & 0 & -1 & 0 \\ 0 & 1 & 2 & 2}\]
\[\dot{x}_1-\dot{x}_3=0\implies \dot{x}_1 = \dot{x}_3\]
\[\dot{x}_2+2\dot{x}_3+2\dot{x}_4=0\implies \dot{x}_2=-2\dot{x}_3-2\dot{x}_4\]
\[ker\mtrx{\cdots}=\Set{\mtrx{\dot{x}_3 \\ -2\dot{x}_3-2\dot{x}_4 \\ \dot{x}_3 \\ \dot{x}_4} \middle\vert \dot{x}_i\in \R}=span\System{\mtrx{1 \\ -2 \\ 1 \\ 0},\mtrx{0 \\ -2 \\ 0 \\ 1}}\]

\item What pair of variables do the equations above not express as functions of the other two? 

\item Is the entire set of $X$ a manifold? 

\begin{proof}
We attempt to find a counterexample, so we want all the columns of 
\[\bm{D}F\Point{\vdots} = \mtrx{2x_1 & 2x_2 & 2x_3 & 2x_4 \\ 1 & 2 & 3 & 4}\]
to be linearly dependent, which means that
\[\frac{x_1}{1}=\frac{x_2}{2}=\frac{-x_3}{3} = \frac{-x_4}{4}\]
whilst still satisfying the original equation
\[x_1^2+x_2^2-x_3^2-x_4^2=0\text{ and }x_1+2x_2+3x_3+4x_4=4\]
There isn't a point where we can solve for $x_i$, so $\bm{D}F$ is always onto, and $X$ is a manifold. 
\end{proof}

\end{enumerate}