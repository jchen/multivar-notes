% !TEX root = ../multivar-notes.tex
\lessondate{February 12, 2019}

\exercise{2.10.9}
Does the system of equations $x+y+\sin (xy)=a$ and $\sin (x^2+y)=2a$ have a solution for sufficiently small $a$?

\answer{Yes}

\begin{proof}
  We rewrite the equation to be $F(x,y,a)=\begin{cases}
    x+y+\sin(xy)-a &= 0 \\
    \sin(x^2+y)-2a &= 0
  \end{cases}$
  \[\bm{DF}(x,y,a) = \mtrx{1 + y\cos(xy) & 1 + x\cos(xy) & -1 \\ 2x\cos(x^2+y) & \cos(x^2+y) & -2}\]
  \[\bm{DF}\vec{c} = \bm{DF}\Point{0 \\ 0 \\ 0} = \mtrx{1 & 1 & -1 \\ 0 & 1 & -2}\]
  $x$ and $y$ are pivotal so $x$ and $y$ both exist.
\end{proof}
\vspace{24pt}

\ul{\textbf{$\bm{k}$-Manifolds in $\R^n$}}

Idea: In BC Calculus, the main object of study was ``functions''. This is too restrictive as many objects that are smooth (have a best linear approximation at each point) are not graphs of functions globally; e.g. circle, spiral, etc.

Since the derivative only tells us about the local properties of a set of points, it suffices to ask that the set is the graph of a diffible function in in some nbhd of every point. A point set in $R^n$ that is locally the graph of some $C^1$-function $\R^k\to \R^{n-k}$ is called a $k$-manifold in $\R^n$.

\begin{defn}
  \textbf{(Smooth manifold in $\R^n$).} A subset $M\subset \R^n$ is a smooth $k$-dimensional manifold if locally it is the graph of a $C^1$ mapping $\bm{f}$ expressing $n-k$ variables as functions of other $k$ variables.
\end{defn}

There are two important ways to define a manifold:
\begin{enumerate}[(i)]
  \item By equation (e.g. $x^2 + y^2 -1 = 0$)
  \item By parametrization: $\displaystyle \gamma (t)=\Point{\cos t \\ \sin t}\quad t\in (0,2\pi)$
\end{enumerate}

\begin{theorem}
  \textbf{(Showing that a locus is a smooth manifold).} (Theorem 3.1.10) ***
\end{theorem}
